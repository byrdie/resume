%%%%%%%%%%%%%%%%%%%%%%%%%%%%%%%%%%%%%%%%%
% Medium Length Professional CV
% LaTeX Template
% Version 2.0 (8/5/13)
%
% This template has been downloaded from:
% http://www.LaTeXTemplates.com
%
% Original author:
% Trey Hunner (http://www.treyhunner.com/)
%
% Important note:
% This template requires the resume.cls file to be in the same directory as the
% .tex file. The resume.cls file provides the resume style used for structuring the
% document.
%
%%%%%%%%%%%%%%%%%%%%%%%%%%%%%%%%%%%%%%%%%

%----------------------------------------------------------------------------------------
%	PACKAGES AND OTHER DOCUMENT CONFIGURATIONS
%----------------------------------------------------------------------------------------

\documentclass{resume} % Use the custom resume.cls style 

\usepackage{fontawesome}
%\usepackage{fbb}
\usepackage[left=0.4 in,top=0.4in,right=0.4 in,bottom=0.4in]{geometry} % Document margins
\usepackage{enumitem}
\usepackage{hyperref}
%\usepackage{natbib}
%\usepackage{bibentry}
\usepackage[style=apa, backend=biber, maxnames=99]{biblatex}
\usepackage{color}

%\renewcommand{\bibsection}{}

\renewcommand*{\mkbibnamegiven}[1]{
	\ifitemannotation{highlight}
	{\textbf{#1}}
	{#1}
}



\renewcommand*{\mkbibnamefamily}[1]{%
	\ifitemannotation{highlight}
	{\textbf{#1}}
	{#1}
}

\newcommand{\tab}[1]{\hspace{.2667\textwidth}\rlap{#1}} 
\newcommand{\itab}[1]{\hspace{0em}\rlap{#1}}

\newcommand{\ceb}[1]{{\color{red} #1}} % Cappy comment

\addbibresource{sources.bib}
%\usepackage{etoolbox}
%\patchcmd{\thebibliography}{\section*{\refname}}{}{}{}

\name{Roy T. Smart} % Your name

\address{Barnard Hall, Room 264, Bozeman, MT 59717}
\address{801-906-1539 \\ roy.smart@montana.edu \\ \faicon{github} \href{github.com/byrdie}{byrdie}}  % Your phone number and email
\address{Department of Physics, Montana State University}



\begin{document}
	
%	\nobibliography*{aassources}
	


%----------------------------------------------------------------------------------------
%	OBJECTIVE
%----------------------------------------------------------------------------------------

%\begin{rSection}{OBJECTIVE}
%
%{``Winners are not people who never fail, but people who never quit". What I've learnt in an assignment is more important to me than what I already knew when I started. By virtue of my tenacity and hard work, I want to build a successful career in mechanical engineering and nanotechnology.}
%
%
%\end{rSection}
%----------------------------------------------------------------------------------------
%	EDUCATION SECTION
%----------------------------------------------------------------------------------------

\begin{rSection}{Interests}
	\begin{minipage}{0.5\textwidth}
		\begin{itemize}[leftmargin=*]
			\item EUV instrumentation (imaging and spectroscopy)
			\item Solar transition region explosive events
			\item Optics modeling and tolerancing
		\end{itemize}
	\end{minipage}
	\begin{minipage}{0.5\textwidth}
		\begin{itemize}[leftmargin=*]
			\item Machine learning
			\item Software engineering
			\item High-performance numerical simulation
		\end{itemize}
	\end{minipage}
\end{rSection} 

\begin{rSection}{Education}

{\bf Ph.D. Candidate in Physics} \hfill {\fontseries{b}\selectfont Aug 2015 - Present}
\\
Department of Physics, Montana State University

{\bf B.S. in Physics, Minor in Computer Science} \hfill {\fontseries{b}\selectfont Aug 2011 - May 2015 }
\\
Department of Physics, Montana State University

\end{rSection}

%----------------------------------------------------------------------------------------
%	WORK EXPERIENCE SECTION
%----------------------------------------------------------------------------------------

\begin{rSection}{WORK EXPERIENCE}
\begin{rSubsection}{Department of Physics, Montana State University}{Jan 2016 - Present}{Graduate Research Assistant for Professor Charles Kankelborg}{Bozeman, MT}
	\small
	\item \textbf{ESIS instrumentation.} The EUV Snapshot Imaging Spectrograph (ESIS) is a NASA sounding rocket mission designed to measure solar transition region spectral line profiles with high spatial, spectral and temporal resolution over a wide field of view.
	\begin{itemize}[topsep=-5pt, noitemsep, leftmargin=*, label={--}]
		\item \textbf{Data analysis.} Implemented a convolutional neural network to recover spectral line profiles from ESIS observations.
		\item \textbf{Optical modeling.} Developed a custom raytrace model in Python to calculate the distortion, vignetting, point-spread function, etc. of the ESIS optical system.
		\item \textbf{Optics testing.} Measured the wavefront error and roughness of the ESIS primary mirror and gratings using phase-shifting interferometry. 
		\item \textbf{Alignment and focus.} Developed procedures to align and focus the ESIS optics using a Zemax model of the instrument.
		\item \textbf{Baffle design}. Developed an automated procedure to generate apertures for the ESIS baffles using a Zemax model of the instrument.
		\item \textbf{Launch campaign.} Operated the ESIS instrument during integration, testing, and flight operations at White Sands Missile Range, NM.
	\end{itemize}
	\item \textbf{FURST optical modeling.} Used Zemax to validate and characterize the optical design for the Full-Sun Ultraviolet Rocket Spectrometer (FURST), a NASA sounding rocket mission designed to measure the vacuum ultraviolet spectrum of the Sun as a star with high resolution.
	\item \textbf{IRIS Planning.} Organized observation schedule for the Interface Region Imaging Spectrograph (IRIS), a NASA Small Explorer mission that observes the solar ultraviolet spectrum with high spatial, spectral, and temporal resolution.
\end{rSubsection}


\begin{rSubsection}{Department of Physics, Montana State University}{Aug 2015 - Dec 2015}{Graduate Teaching Assistant for Professor Nicholas Childs}{Bozeman, MT}
	\small
	\item \textbf{Undergradute Physics I.} Assisted with conducting the labs, test proctoring, and grading.
\end{rSubsection}

\begin{rSubsection}{Department of Physics, Montana State University}{Sep 2012 - Aug 2015}{Undergraduate Research Assistant for Professor Charles Kankelborg}{Bozeman, MT}
	\small
	\item \textbf{MOSES instrumentation.} The Mulit-Order Solar EUV Spectrograph (MOSES) is the predecessor to ESIS and is also a NASA sounding rocket mission designed to measure solar transition region spectral line profiles with high spatial, spectral and temporal resolution over a wide field of view.
	\begin{itemize}[topsep=-5pt, noitemsep, leftmargin=*, label={--}]
		\item \textbf{Flight software.} Developed and tested software to control the cameras on MOSES and downlink the images.
		\item \textbf{Data analysis.} Developed a convolutional neural network to process observations from the MOSES instrument.
		\item \textbf{Launch campaign.} Operated the MOSES instrument during integration, testing, and flight operations at White Sands Missile Range, NM.
	\end{itemize}
\end{rSubsection}

%\begin{rSubsection}{Colorado River \& Trail Expeditions}{June 2011 - Aug 2013}{Whitewater River Guide}{Salt Lake City, UT}
%	\small
%	\item \textbf{Grand Canyon River Guide.} Piloted 18' rafts through the Grand Canyon on trips lasting up to 12 days.
%	Navigated Class V rapids, entertained guests, and prepared meals while working as a team.
%	\item \textbf{Trip planning.} Prepared menu, purchased supplies, organized logistics, and maintained equipment for upcoming trips.
%\end{rSubsection}

\newpage
	
\end{rSection} 

%----------------------------------------------------------------------------------------
%	TECHNICAL STRENGTHS SECTION
%----------------------------------------------------------------------------------------

\begin{rSection}{TECHNOLOGY SKILLS}

\begin{tabular}{ @{} >{\bfseries}l @{\hspace{6ex}} l }
Design Software & Zemax \\ 
%Engineering Software & Abaqus, MATLAB, Ansys Mechanical APDL, COMSOL Multiphysics.  \\  
Programming Languages &  Python, C/C++, IDL, \LaTeX, CUDA C/C++, Java, Mathematica \\
Frameworks and libraries & Numpy, Scipy, Astropy, Sunpy, Numba, Pandas, Keras, Sphinx \\
Office Accessories & Word, Powerpoint, Excel, Project, Visio
\end{tabular}

\end{rSection}
\label{key}




%	EXAMPLE SECTION
%----------------------------------------------------------------------------------------

\begin{rSection}{Awards}
\textbf{NASA Earth and Space Science Fellowship (NESSF)} \hfill {\fontseries{b}\selectfont Sep 2017 - Sep 2020} \\
{``Neural Networks for Computed Tomography Imaging Spectroscopy''}

\end{rSection}

\begin{rSection}{PUBLICATIONS}
	\small
	\begin{itemize}[leftmargin=*]
		\item \fullcite{Parker2022}
		\item \fullcite{Smart2022}
	\end{itemize}
	
\end{rSection}

\begin{rSection}{PRESENTATIONS}
\small
\begin{itemize}[leftmargin=*]
	\item \fullcite{2020AGUFMSH0480003S}
	\item \fullcite{2020AGUFMSH0480004P}
	\item \fullcite{2019AGUFMSH31C3321S}
	\item \fullcite{2019AGUFMSH33A..05K}
	\item \fullcite{2018AGUFMSH23A..05S}
	\item \fullcite{2018SPIE10747E..0AJ}
	\item \fullcite{2017SPD....4810610S}
	\item \fullcite{2017SPD....4811001K}
	\item \fullcite{2016SPD....4730901S}
\end{itemize}

%\printbibliography

%\nocite{*}\left( 
%\bibliographystyle{aasjournal}
%\bibliography{aassources}

\end{rSection}

%----------------------------------------------------------------------------------------

%		REFERENCE SECTION
%----------------------------------------------------------------------------------------

\begin{rSection}{REFERENCES}
\begin{minipage}[h]{0.5\textwidth}
Dr. Charles Kankelborg\\
Professor\\
Department of Physics, Montana State University\\Bozeman, MT\\
Tel: 406-994-7853\\
Email: kankel@montana.edu\\
Relation: Supervisor \& course teacher 
\end{minipage}
\hfill
\begin{minipage}[h]{0.5\textwidth}
Dr. Jacob Parker\\
Research Astrophysicist\\
NASA Goddard Space Flight Center\\ Greenbelt, MD\\
Tel: 208-520-2807 (cell)\\
Email: jacob.d.parker@nasa.gov\\
Relation: Colleague
\end{minipage}
\hfill
\end{rSection}

\begin{rSection}{PERSONAL INTERESTS}
	\small
	\begin{minipage}{0.17\textwidth}
		\begin{itemize}[leftmargin=*]
			\item Alpine skiing
		\end{itemize}
	\end{minipage}
	\begin{minipage}{0.23\textwidth}
		\begin{itemize}[leftmargin=*]
			\item Whitewater rafting	
		\end{itemize}
	\end{minipage}
	\begin{minipage}{0.35\textwidth}
		\begin{itemize}[leftmargin=*]
			\item Open-source software development
		\end{itemize}
	\end{minipage}
	\begin{minipage}{0.2\textwidth}
		\begin{itemize}[leftmargin=*]
			\item Amateur astronomy
		\end{itemize}
	\end{minipage}
\end{rSection}

\end{document}