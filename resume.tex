%%%%%%%%%%%%%%%%%%%%%%%%%%%%%%%%%%%%%%%%%
% Medium Length Professional CV
% LaTeX Template
% Version 2.0 (8/5/13)
%
% This template has been downloaded from:
% http://www.LaTeXTemplates.com
%
% Original author:
% Trey Hunner (http://www.treyhunner.com/)
%
% Important note:
% This template requires the resume.cls file to be in the same directory as the
% .tex file. The resume.cls file provides the resume style used for structuring the
% document.
%
%%%%%%%%%%%%%%%%%%%%%%%%%%%%%%%%%%%%%%%%%

%----------------------------------------------------------------------------------------
%	PACKAGES AND OTHER DOCUMENT CONFIGURATIONS
%----------------------------------------------------------------------------------------

\documentclass{resume} % Use the custom resume.cls style

\usepackage[left=0.4 in,top=0.4in,right=0.4 in,bottom=0.4in]{geometry} % Document margins
\usepackage{enumitem}
\usepackage[style=numeric, backend=biber, maxnames=99]{biblatex}
\newcommand{\tab}[1]{\hspace{.2667\textwidth}\rlap{#1}} 
\newcommand{\itab}[1]{\hspace{0em}\rlap{#1}}

\addbibresource{sources.bib}
%\usepackage{etoolbox}
%\patchcmd{\thebibliography}{\section*{\refname}}{}{}{}

\name{Roy T. Smart} % Your name
\address{Department of Physics, Montana State University}
\address{Barnard Hall, Room 264}
\address{+1 (801)-906-1539 \\ roy.smart@montana.edu}  % Your phone number and email

\begin{document}

%----------------------------------------------------------------------------------------
%	OBJECTIVE
%----------------------------------------------------------------------------------------

%\begin{rSection}{OBJECTIVE}
%
%{``Winners are not people who never fail, but people who never quit". What I've learnt in an assignment is more important to me than what I already knew when I started. By virtue of my tenacity and hard work, I want to build a successful career in mechanical engineering and nanotechnology.}
%
%
%\end{rSection}
%----------------------------------------------------------------------------------------
%	EDUCATION SECTION
%----------------------------------------------------------------------------------------

\begin{rSection}{Interests}
	\begin{minipage}{0.5\textwidth}
		\begin{itemize}[leftmargin=*]
			\item Solar imaging and spectroscopy
			\item Solar transition region explosive events
			\item Optics modeling and tolerancing
		\end{itemize}
	\end{minipage}
	\begin{minipage}{0.5\textwidth}
		\begin{itemize}[leftmargin=*]
			\item Machine learning
			\item Software engineering
			\item High-performance numerical simulation
		\end{itemize}
	\end{minipage}
\end{rSection} 

\begin{rSection}{Education}

{\bf Ph.D. Candidate in Physics} \hfill { Aug 2015 - Present}
\\
Department of Physics, Montana State University

{\bf Bachelor of Science in Physics} \hfill { Aug 2011 - May 2015 }
\\
Department of Physics, Montana State University

\end{rSection}

%----------------------------------------------------------------------------------------
%	WORK EXPERIENCE SECTION
%----------------------------------------------------------------------------------------

\begin{rSection}{WORK EXPERIENCE}
\begin{rSubsection}{Department of Physics, Montana State University}{Jan 2016 - Present}{Graduate Research Assistant for Professor Charles Kankelborg}{Bozeman, MT}
	\item ESIS data analysis. Developed a convolutional neural network algorithm to process observations from the EUV Snapshot Imaging Spectrograph (ESIS), a NASA sounding rocket mission designed to measure the velocity of plasma in the solar transition region.
	\item ESIS Python library. Optical parameters and raytrace model of the ESIS instrument.
	\item Optical testing and assembly of ESIS. Checked quality of optics using phase-shifting interferometry. Developed procedures to align and focus the ESIS optics using a Zemax model of the instrument.
	\item ESIS launch campaign at White Sands Missile Range, NM. Operated the ESIS instrument during integration, testing, and flight operations.
	\item Optical design of the Full-Sun Ultraviolet Rocket Spectrometer (FURST), a NASA sounding rocket mission designed to measure the vacuum ultraviolet spectrum of the Sun as a star. Used Zemax to validate and characterize an optical system proposed by Charles Kankelborg.
\end{rSubsection}


\begin{rSubsection}{Department of Physics, Montana State University}{Aug 2015 - Dec 2015}{Graduate Teaching Assistant for Professor Nicholas Childs}{Bozeman, MT}
	\item Assisted with conducting the labs, test proctoring, and grading.
\end{rSubsection}

\begin{rSubsection}{Department of Physics, Montana State University}{Sep 2012 - Aug 2015}{Undergraduate Research Assistant for Professor Charles Kankelborg}{Bozeman, MT}
	\item MOSES II flight software development. Wrote and tested software to control the cameras on the Multi-Order Solar EUV Spectrograph (MOSES), a predecessor to ESIS which also measures the velocity of plasma in the solar transition region. \\
	\item MOSES data analysis. Trained a convolutional neural network to process observations from the MOSES instrument. \\
	\item MOSES II launch campaign. Operated the MOSES instrument during integration, testing, and flight operations at White Sands Missile Range, NM.
\end{rSubsection}

%\sl\textbf{Graduate Research Assistant} \hfill Jan 2016 - Present \\
%Professor Charles Kankelborg, Department of Physics, Montana State University \\
%{\textbullet ESIS data analysis. Developed a convolutional neural network algorithm to process observations from the EUV Snapshot Imaging Spectrograph (ESIS), a NASA sounding rocket mission designed to measure the velocity of plasma in the solar transition region.}\\
%{\textbullet ESIS Python library. Optical parameters and raytrace model of the ESIS instrument.}\\
%{\textbullet Optical testing and assembly of ESIS. Checked quality of optics using phase-shifting interferometry. Developed procedures to align and focus the ESIS optics using a Zemax model of the instrument.}\\
%{\textbullet ESIS launch campaign at White Sands Missile Range, NM. Operated the ESIS instrument during integration, testing, and flight operations.} \\
%{\textbullet Optical design of the Full-Sun Ultraviolet Rocket Spectrometer (FURST), a NASA sounding rocket mission designed to measure the vacuum ultraviolet spectrum of the Sun as a star. Used Zemax to validate and characterize an optical system proposed by Charles Kankelborg.} \\
%
%\sl\textbf{Graduate Teaching Assistant for Introductory Physics I} \hfill Aug 2015 - Dec 2015 \\
%Professor Nicholas Childs, Department of Physics, Montana State University \\
%{\textbullet Assisted with conducting the labs, test proctoring, and grading} \\
%
%\sl\textbf{Undergraduate Research Assistant} \hfill Sep 2012 - Aug 2015 \\
%Professor Charles Kankelborg, Department of Physics, Montana State University \\
%{\textbullet MOSES II flight software development. Wrote and tested software to control the cameras on the Multi-Order Solar EUV Spectrograph (MOSES), a predecessor to ESIS which also measures the velocity of plasma in the solar transition region.} \\
%{\textbullet MOSES data analysis. Trained a convolutional neural network to process observations from the MOSES instrument.} \\
%{\textbullet MOSES II launch campaign. Operated the MOSES instrument during integration, testing, and flight operations
%at White Sands Missile Range, NM.} \\
	
\end{rSection} 

%----------------------------------------------------------------------------------------
%	TECHNICAL STRENGTHS SECTION
%----------------------------------------------------------------------------------------

\begin{rSection}{TECHNOLOGY SKILLS}

\begin{tabular}{ @{} >{\bfseries}l @{\hspace{6ex}} l }
Design Software & Zemax \\ 
%Engineering Software & Abaqus, MATLAB, Ansys Mechanical APDL, COMSOL Multiphysics.  \\  
Programming Languages &  Python, C/C++, IDL, CUDA C/C++, Java, Mathematica \\
Frameworks and libraries & Numpy, Scipy, Astropy, Pandas, Keras, Sphinx \\
Office Accessories & MS Word, MS Power-point, MS Excel etc.\\
\end{tabular}

\end{rSection}





%	EXAMPLE SECTION
%----------------------------------------------------------------------------------------

\begin{rSection}{Awards}
\textbf{NASA Earth and Space Science Fellowship} \hfill Sep 2017 - Sep 2020 \\
%{\textbullet Proposed a project titled ``Neural Networks for Computed Tomography Imaging Spectroscopy''}

\end{rSection} 

\begin{rSection}{PUBLICATIONS AND PRESENTATIONS}

%\nocite{*}
\begin{itemize}[leftmargin=*]
	\item \fullcite{2020AGUFMSH0480004P}
	\item \fullcite{2020AGUFMSH0480003S}
	\item \fullcite{2019AGUFMSH33A..05K}
	\item \fullcite{2019AGUFMSH31C3321S}
	\item \fullcite{2018AGUFMSH23A..05S}
	\item \fullcite{2018SPIE10747E..0AJ}
	\item \fullcite{2017SPD....4811001K}
	\item \fullcite{2017SPD....4810610S}
	\item \fullcite{2016SPD....4730901S}
\end{itemize}
%\printbibliography

%\bibliographystyle{plainyr-rev}


\end{rSection}

%----------------------------------------------------------------------------------------

%		REFERENCE SECTION
%----------------------------------------------------------------------------------------

\begin{rSection}{REFERENCES}
\begin{minipage}[h]{0.5\textwidth}
{\sl Dr. Charles Kankelborg\\
Professor\\
Department of Physics, Montana State University,\\Bozeman, MT\\
Tel: 406-994-7853\\
email: kankel@montana.edu\\
Relation: Supervisor \& Course Teacher} 
\end{minipage}
\hfill
\begin{minipage}[h]{0.5\textwidth}
{\sl Dr. Jacob Parker\\
Research Scientist\\
NASA Goddard Space Flight Center,\\ Greenbelt, MD\\
Tel: 208-520-2807 (Cell)\\
email: jacob.d.parker@nasa.gov\\
Relation: Colleague} 
\end{minipage}
\hfill
\end{rSection}
\end{document}